 1. Crie uma classe em Java chamada fatura para uma loja de
suprimentos de informática. A classe deve conter quatro variáveis – o número (String), a
descrição (String), a quantidade comprada de um item (int) e o preço por item
(double). A classe deve ter um construtor e um método get e set para cada variável de
instância. Além disso, forneça um método chamado getTotalFatura que calcula o valor
da fatura e depois retorna o valor como um double. Se o valor não for positivo, ele deve ser
configurado como 0. Se o preço por item não for positivo, ele deve ser configurado como 0.0.
Escreva um aplicativo de teste chamado FaturaTeste (em outro arquivo) que demonstra
as capacidades da classe Fatura. 